\documentclass{beamer}
\usepackage{pgf}
\usepackage{amsmath}
\usepackage{multirow}
\usepackage{setspace}
\usepackage{threeparttable}
\usepackage[english]{babel}
\usepackage{multimedia}
\usepackage{hyperref}
\usepackage{epstopdf}
\usepackage{subfigure}
\usepackage{graphicx}
\usepackage{scalefnt}
\usepackage{tikz}
\usepackage{eepic}

\DeclareMathOperator*{\plim}{plim}

\def\independenT#1#2{\mathrel{\rlap{$#1#2$}\mkern2mu{#1#2}}}

\mode<presentation>
{
\setbeamercovered{transparent}
\usetheme{CambridgeUS}
\usecolortheme{dolphin}
}
\setbeamertemplate{button}{\tikz
  \node[
  inner xsep=10pt,
  draw=structure!80,
  fill=structure!50,
  rounded corners=4pt]  {\Large\insertbuttontext};}

\setbeamertemplate{section page}
{
    \begin{centering}
    \begin{beamercolorbox}[sep=12pt,center]{part title}
    \usebeamerfont{section title}\insertsection\par
    \end{beamercolorbox}
    \end{centering}
}

\title{Human Capital and Signaling}

\begin{document}
\maketitle
\section{Introduction}
\frame{\sectionpage}
\begin{frame}[<+->]{Introduction}
    \begin{itemize}
        \item Why do high school students desire to go to the Ivy League?
        \item Why do employers want to hire Ivy league graduates?
        \item \textbf{Question of the week}: Do we go to school (or training) to learn new skills or does schooling separate those with higher ability from those with lower ability? 
        \item OR IS BOTH!?! ... we shall find out (or maybe not). 
    \end{itemize}
    
\end{frame}


\section{Human Capital Model}
\frame{\sectionpage}
\begin{frame}[<+->]{Basics of Human Capital}
	\begin{itemize}
	    \item What separates human capital from other forms of capital?
	
    	\begin{itemize}
    	    \item Human capital cannot be \textbf{collateralized}, meaning that it is not a physical asset that can be seized by a lender if a loan is not paid back.
    	    \item Human capital cannot be owned by anyone other than the individual and cannot be sold.
    	\end{itemize}
        \item Human capital includes more than just formal education (e.g. athletic or musical talent).
        \item Broadly it covers the skills, knowledge, and attributes of a worker that have value in the labor market.
    \end{itemize}
\end{frame}

\begin{frame}[<+->]{Human Capital Model}

    \begin{itemize}
        \item Basic Costs
        \begin{itemize}
            \item Direct costs (e.g. tuition, interest on student loans)
            \item Foregone wages
        \end{itemize}
        \item Basic Benefits 
        \begin{itemize}
           \item Increased wages (decreasing in age)
           \item non-monetary life improvements
       \end{itemize}
        \begin{figure}[h]
   			 \includegraphics<+->[width = 2.25in]{turner1e_fig_04_01.png}
   		\end{figure}
    \end{itemize}
    
\end{frame}

\begin{frame}[<+->]{Predictions from the HC Model}
    \begin{itemize}
        \item Demand for formal education is positively correlated with the college wage premium (and demand for skilled workers). 
        \item There are diminishing marginal returns to education 
        \item People will consume education until marginal rate of return equals the marginal cost of capital
        \begin{figure}[h]
   			 \includegraphics<+->[width = 2.25in]{turner1e_fig_04_04.png}
   		\end{figure}    
    \end{itemize}
    
\end{frame}

\section{Signaling Theory}
\frame{\sectionpage}

\begin{frame}[<+->]{Signaling Theory Setup}
     \begin{itemize}
        \item The labor market has information asymmetries as employers don't have full information about the potential quality of new hires.
        \item Therefore employers look for markers or signals of quality
        \begin{itemize}
            \item Educational attainment
            \item Selectivity of undergraduate institution 
            \item Law school review
        \end{itemize}
        \item Intuition is that only individuals with the desired skills have the ability to complete the "signal" 
        \begin{itemize}
            \item Higher ability individuals see the better returns to certain fields/careers
            \item Pursuing signals are less costly for them
        \end{itemize}
        \item The signal does not represent skills and knowledge acquired during the learning process.
    \end{itemize}   
\end{frame}

\begin{frame}{Signaling Equilibrium}
    \begin{figure}[h]
   		 \includegraphics<+->[width = 2.25in]{turner1e_fig_05_01.png}
   	\end{figure} 
\end{frame}

\begin{frame}[<+->]{Separating and Pooling Equilibria}
    \begin{itemize}
        \item A \textbf{separating equilibrium} exists when 
        \begin{itemize}
            \item education (\textit{e}) can be used to distinguish high (\textit{H}) and low (\textit{L}) ability
            \item employees can then pay \textit{H} workers a higher wage $w_H$ and \textit{L} workers a lower wage $w_L$
        \end{itemize}
        \item a \textbf{pooling equilibrium} exists when the cost of \textit{e} is low enough that both \textit{H} and \textit{L} workers can afford it
        \begin{itemize}
            \item Employers therefore pay a wage that is a weighted average of type \textit{H} and type \textit{L} workers in the field
            \item Why do teacher  bonuses for Masters degrees create a pooling equilibrium?
        \end{itemize}
    
    \end{itemize}
\end{frame}


\begin{frame}{Differences between the Theories}
    Why is it important for education policy to differentiate between the human capital and signaling models? \medskip

    The models differ on two key dimensions
    \begin{itemize}
        \item All returns in the signaling model are private. In the HC model society receives some benefit.
        \item In the HC model, how we arrange inputs and/or design the education process matters. In signaling it does not.
    \end{itemize}
    \medskip
    Due to the positive externalities of education in the HC model, it is useful for governments to intervene. Why is this not the case under the signaling model? 
    \medskip
    
    Important to keep in mind that in most cases reality is a blend of both models.
    
\end{frame}

\begin{frame}[<+->]{Human Capital or Signaling}
    \begin{itemize}
        \item The National Board Certification for teachers?
        \item A college degree versus no college degree?
        \item An economics PhD versus an undergraduate degree in economics?
        \item A Harvard economics PhD versus a Penn State Economics PhD?
    \end{itemize}
    
\end{frame}

\section{Regression Discontinuity Designs}
\frame{\sectionpage}

\begin{frame}[<+->]
	\begin{itemize}
		\item Along with RCTs, RDs have become the gold standard of education and social policy research
		\item When done correctly, they provide an internally valid estimate of a policy/practice/treatment, but there are external validity costs
		\item RDs rely on a decision rule where treatment is controlled by a person's (or some other unit) value on a forcing variable \medskip
		\begin{itemize}
			\item Math score on a placement exam for remedial math
			\item Income for post-secondary financial aid
			\item SAT cutoffs for university admissions
		\end{itemize}
			\item A comparison of individuals' outcomes just below/above the treatment cutoff will provide an unbiased treatment effect estimate
			\item RD comes in two flavors: Sharp and Fuzzy
			\begin{itemize}
				\item Sharp=A setting where treatment is 100\% dictated by your score on the forcing variable
				\item Fuzzy=A setting where there's a jump in the likelihood of treatment at a threshold, but your forcing variable score does not perfectly determine treatment (e.g. non-compliance)
			\end{itemize}
		\end{itemize}

\end{frame}


\begin{frame}
\begin{columns}
	\begin{column}{0.48\textwidth}
  		  \begin{figure}[h]
   			 \includegraphics[width = 2.15in]{turner1e_fig_03_04.png}
   		 \end{figure}
	\end{column}
	\begin{column}{0.48\textwidth}
		\begin{itemize}[<+->]
			\item In Figure A, we see there is a 75 percentage point difference in the likelihood of winning a merit scholarship at a test score cutoff (fuzzy cutoff) \bigskip
			\item We can use this fuzzy cutoff to estimate the effect of merit aid on graduating college
		\end{itemize}
		
	\end{column}
\end{columns}

\end{frame}


\begin{frame}
\begin{columns}
	\begin{column}{0.48\textwidth}
  		  \begin{figure}[h]
   			 \includegraphics[width = 2.15in]{turner1e_fig_03_04.png}
   		 \end{figure}
	\end{column}
	\begin{column}{0.48\textwidth}
		\begin{itemize}[<+->]
			\item In a fuzzy framework, we divide the difference in outcomes at the threshold (10 percentage point increase in graduating from college (or .1)) by the difference in the likelihood of treatment (75 percentage points (or .75)) \medskip
			\item Here we find a 13.3 percentage point increase in the likelihood of graduating from college due to merit based financial aid
		\end{itemize}
		
	\end{column}
\end{columns}

\end{frame}












\end{document}




