\documentclass{beamer}
\usepackage{pgf}
\usepackage{amsmath}
\usepackage{multirow}
\usepackage{setspace}
\usepackage{threeparttable}
\usepackage[english]{babel}
\usepackage{multimedia}
\usepackage{hyperref}
\usepackage{epstopdf}
\usepackage{subfigure}
\usepackage{graphicx}
\usepackage{scalefnt}
\usepackage{tikz}
\usepackage{eepic}

\DeclareMathOperator*{\plim}{plim}

\def\independenT#1#2{\mathrel{\rlap{$#1#2$}\mkern2mu{#1#2}}}

\mode<presentation>
{
\setbeamercovered{transparent}
\usetheme{CambridgeUS}
\usecolortheme{dolphin}
}
\setbeamertemplate{button}{\tikz
  \node[
  inner xsep=10pt,
  draw=structure!80,
  fill=structure!50,
  rounded corners=4pt]  {\Large\insertbuttontext};}

\setbeamertemplate{section page}
{
    \begin{centering}
    \begin{beamercolorbox}[sep=12pt,center]{part title}
    \usebeamerfont{section title}\insertsection\par
    \end{beamercolorbox}
    \end{centering}
}

\title{School Accountability}

\begin{document}
\maketitle

\begin{frame}[<+->]{Background of Accountability}
	\begin{itemize}
		\item There are many types of accountability in public education \\~\\
		\item Examples include:
			\begin{enumerate}
				\item Democratic processes: e.g. school board elections
				\item Parental/consumer pressure: tiebout sorting, school choice
				\item Performance based: use of incentives, etc.   \\~\\
			\end{enumerate}
		\item We'll focus on school accountability policies (e.g. performance-based accountability) today, and some of market driven accountability in the school choice week. 
	\end{itemize}
\end{frame}		


\begin{frame}[<+->]{Background of Accountability}
	\begin{itemize}
		\item The accountability movement that we now know came out of the Standards Based Reform movement of the 1990s
		\item This movement had many goals, one of which was to improve school accountability:
			\begin{enumerate}
				\item Clear academic expectations
				\item Alignment of key state education policies
				\item Use of assessments to measure student outcomes
				\item Decentralization of resources, curriculum, and instruction
				\item Technical assistance for failing schools 
				\item Use of \textbf{accountability policies} to reward and sanction high/low performing schools
			\end{enumerate}
		\item In this movement, accountability was the last thing you put in place once the first 5 items were well established. How does that compare to how accountability policies are implemented today?
	\end{itemize}
\end{frame}

\begin{frame}[<+->]{Background of Accountability}
	\begin{itemize}
		\item Performance-based accountability polices are usually motivated by the principal-agent model
		\begin{itemize}
			\item What problem does the PA model to solve?
			\item In education, who are the ``principals'' and who are the ``agents''?
			\item What assumptions are needed for the PA model to work as intended? \\~\\
		\end{itemize}
		
		\item Most accountability systems hold schools accountable for test scores, graduation rates, and maybe a handful of other outcomes
		\item They often don't provide a lot of information about what schools are and are not doing well
		\item In this case, what is a big assumption with how accountability policies will get educators to improve student outcomes?
	\end{itemize}
\end{frame}


\begin{frame}[<+->]{Background of Accountability}
	\begin{itemize}
		\item Performance-based accountability polices are usually motivated by the principal-agent model
		\item What problem does the PA model to solve?
		\item In education, who are the ``principals'' and who are the ``agents''?
		\item What assumptions are needed for the PA model to work as intended?
	\end{itemize}
\end{frame}


\begin{frame}[<+->]{Background of Accountability}
	\begin{itemize}
		\item Accountability policies are prone to the classic multi-tasking problem
		\item While goals of education are multidimensional, we often hold schools accountable for at best a handful of outcomes that hopefully proxy these goals. 
		\item What happens to outcomes that we care about but are not measured in accountability systems?
		\item These policies are also prone to other maladaptive behaviors:
			\begin{itemize}
				\item cheating
				\item narrowing of curriculum and/or teaching to the test
				\item focus on bubble kids
				\item push out low performing students
				\item and so on
			\end{itemize} 
	\end{itemize}
\end{frame}


\begin{frame}[<+->]{Does accountability work?}
	\begin{itemize}
		\item It depends ... [when isn't this the answer though?]
		\item Early policies that did not attach stakes (e.g. rewards or sanctions) appear to not have changed behavior
		\item We have evidence from three different governance levels:
			\begin{enumerate}
				\item Federal
				\item State
				\item District
			\end{enumerate}
		\item As with most topics, this is an area where there are important general versus partial equilibrium considerations
	\end{itemize}
	
\end{frame}


\end{document}