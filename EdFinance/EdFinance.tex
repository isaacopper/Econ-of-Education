\documentclass{beamer}
\usepackage{pgf}
\usepackage{amsmath}
\usepackage{multirow}
\usepackage{setspace}
\usepackage{threeparttable}
\usepackage[english]{babel}
\usepackage{multimedia}
\usepackage{hyperref}
\usepackage{epstopdf}
\usepackage{subfigure}
\usepackage{graphicx}
\usepackage{scalefnt}
\usepackage{tikz}
\usepackage{eepic}

\DeclareMathOperator*{\plim}{plim}

\def\independenT#1#2{\mathrel{\rlap{$#1#2$}\mkern2mu{#1#2}}}

\mode<presentation>
{
\setbeamercovered{transparent}
\usetheme{CambridgeUS}
\usecolortheme{dolphin}
}
\setbeamertemplate{button}{\tikz
  \node[
  inner xsep=10pt,
  draw=structure!80,
  fill=structure!50,
  rounded corners=4pt]  {\Large\insertbuttontext};}

\setbeamertemplate{section page}
{
    \begin{centering}
    \begin{beamercolorbox}[sep=12pt,center]{part title}
    \usebeamerfont{section title}\insertsection\par
    \end{beamercolorbox}
    \end{centering}
}

\title{Education Finance}

\begin{document}
\maketitle

\section{Introduction}

\begin{frame}{Changes in Education Funding}
\begin{itemize}
\item Initially, all education policy was decentralized. 
	\begin{itemize}
	\item Lots of small school districts. 
	\item Almost all funding came from local sources. 
	\item Education was pretty cheap.
	\end{itemize}
\item In 1940:
	\begin{itemize}
	\item There were $\sim 120,000$ school districts.
	\item Per-public expenditures were roughly $\$2,000$ (in 2011 terms).
	\item Almost $70\%$ that came from local sources.
	\end{itemize}
\item By 2010:
	\begin{itemize}
	\item There were $\sim 13,000$ school districts.
	\item Per-public expenditures were roughly $\$12,000$ (in 2011 terms).
	\item About $40\%$ that came from local sources, about $40\%$ from the state, and about $10\%$.
	\end{itemize}
\pause
\item \textbf{Question:} How did that happen? What caused the changes?
\end{itemize}
\end{frame}

\begin{frame}{Two Major Changes}
\begin{itemize}
\item Elementary and Secondary Education Act (ESEA)
	\begin{itemize}
	\item Part of President Lyndon B. Johnson's ``War on Poverty"
	\item Title I provides funding for low-income students. 
	\item Increased the federal governments involvement in education and gave it a way to add incentives to schools. 
		\begin{itemize}
		\item For example, the No Child Left Behind made the receipt of Title I funds conditional on schools having some form of accountability and required states that received Title I funds to give assessments. \\~\\
		\end{itemize}
	\end{itemize}
\item School Finance Equalizations
	\begin{itemize}
	\item Local financing of public schools almost always leads to more inequality. 
	\item Beginning in the 1970s, numerous state lawsuits argued that the inequality was not constitutional according to state constitutions. 
	\item This lead to lots of attempts to centralize school financing. 
		\begin{itemize}
		\item States varied in the ways they centralized school financing; some did it well and some did not. \\~\\
		\end{itemize}
	\end{itemize}
\end{itemize}
\end{frame}

\begin{frame}{California's History}
\begin{itemize}
\item Serrano v Priest (1971)
	\begin{itemize}
	\item Education is a fundamental constitutional right and that the inequities produced by local financing violates the equal protection clause of the state constitution. 
	\item In response, the state legislature passed SB 90 which established a new funding scheme combining state aid and local property taxes to increase parity. \\~\\
	\end{itemize}
\item Prop 13 (1978)
	\begin{itemize}
	\item Proposition 13 limited local property taxes to 2\% annual increases, leading to much lower revenues for local districts. 
	\item In response, the state legislature passed AB 8 which shifted lots of tax revenue to local governments and created a state fund to support schools. 
	\end{itemize}
\end{itemize}
\end{frame}

\begin{frame}{California's History}
\begin{itemize}
\item Authorization of the State Lottery (1984)
	\begin{itemize}
	\item In California, like most states, a large chunk of the funds (34\%) collected from the lottery go to schools. \\~\\
	\end{itemize} 
\item Prop 98 (1988)
	\begin{itemize}
	\item Provides a minimum funding guarantee for school districts. 
	\item The minimum guarantee depends on the amount of state revenue, so is quite volatile. \\~\\
	\end{itemize}
\item Prop 30 (2012)
	\begin{itemize}
	\item Increases state income taxes on high earners, with the revenue in part going to fund public schools.
	\end{itemize}
\end{itemize}
\end{frame}

\begin{frame}{LAUSD Funding}
\begin{itemize}
\item Program Revenue: $\$2.1$ million
	\begin{itemize}
	\item $\$1.8$ million of this comes from ``operating grants and contributions" \\~\\
	\end{itemize}
\item General Revenue $\$6.7$ million
	\begin{itemize}
	\item Property Taxes: $\$2.4$ million
	\item State Aid: $\$3.9$ million \\~\\
	\end{itemize}
\item Overall:
	\begin{itemize}
	\item $44\%$ of funding from state aid
	\item $27\%$ of funding from property taxes
	\item $21\%$ of funding from operating grants and contributions
	\item $8\%$ of funding from other sources
	\end{itemize}
\end{itemize}
\end{frame}

\begin{frame}{LAUSD Expenditures}
\begin{itemize}
\item Instruction: $\$4.5$ million
\item Student Support Services: $\$460,000$
\item Instructional Staff Support Services: $\$584,000$
\item School Administration Support Services: $\$512,000$
\item Operation/Maintenance of Plant Services: $\$780,000$
\item Student Transportation: $\$186,000$
\item Non-instructional services: $\$528,000$
\end{itemize}
\end{frame}

\end{document}