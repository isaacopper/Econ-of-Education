\documentclass{beamer}
\usepackage{pgf}
\usepackage{amsmath}
\usepackage{multirow}
\usepackage{setspace}
\usepackage{threeparttable}
\usepackage[english]{babel}
\usepackage{multimedia}
\usepackage{hyperref}
\usepackage{epstopdf}
\usepackage{subfigure}
\usepackage{graphicx}
\usepackage{scalefnt}
\usepackage{tikz}
\usepackage{eepic}

\DeclareMathOperator*{\plim}{plim}

\def\independenT#1#2{\mathrel{\rlap{$#1#2$}\mkern2mu{#1#2}}}

\mode<presentation>
{
\setbeamercovered{transparent}
\usetheme{CambridgeUS}
\usecolortheme{dolphin}
}
\setbeamertemplate{button}{\tikz
  \node[
  inner xsep=10pt,
  draw=structure!80,
  fill=structure!50,
  rounded corners=4pt]  {\Large\insertbuttontext};}

\setbeamertemplate{section page}
{
    \begin{centering}
    \begin{beamercolorbox}[sep=12pt,center]{part title}
    \usebeamerfont{section title}\insertsection\par
    \end{beamercolorbox}
    \end{centering}
}

\title{Returns to Education}

\begin{document}
\maketitle

\section{Introduction}

\begin{frame}{Markets in Education}
Education is a unique marketplace:
	\begin{itemize}
	\item It is not perfectly competitive (e.g. large fixed costs, imperfect information);
	\item Spillovers or externalities are quite common;
	\item Supply is often local(ish);
	\item Education is highly differentiated, both vertically (age) and horizontally (different quality/classes for individuals who are the same age);
	\item The quality of your education (might) depend on the characteristics of other consumers (i.e. your peers).
	\end{itemize}
That means that we can use economic tools to answer questions about education policy, but we need to go beyond the simplest models of perfect competition. Although plenty of policy-makers and commentators focus on trying to get the market more toward their perfect competition ideal.
\end{frame}

\begin{frame}[<+->]{Education as an Investment}
\begin{itemize}
	\item Simple economic model:
	\begin{itemize}
	\item An individual's utility depends on their consumption: $u(c)$
	\item Their consumption depends on their wage: $p \cdot c \leq w$
	\item Their wage depends on their eduction: $w = w(e)$
	\end{itemize}
\item One of the earliest (and most important) questions in the economic of eduction: what are the returns to eduction?
	\begin{itemize}
	\item For now, we will focus on the individual returns to eduction as opposed to the social returns to eduction.  \\~\\
	\end{itemize}
\item What do we mean by the ``returns to eduction?" \\~\\
\item Why do we care about the individual returns to eduction?
\end{itemize}
\end{frame}

\begin{frame}[<+->]{Estimating the Individual Returns to Eduction}
\begin{itemize}
\item Let's suppose that you ran an OLS regression:
	\begin{equation}
	ln(Y_i) = \beta_0 + \beta_1 Education_i + \epsilon_i
	\end{equation}
\item Does $\beta_1$ define the individual returns to education? \\~\\
\item If not, is $\beta_1$ larger or smaller than the returns to eduction?
\end{itemize}
\end{frame}

\begin{frame}{Omitted Variable Bias}
\begin{itemize}
		\item Let's assume that an individual's outcome $Y_{i}$ is a function of her years of education ($Ed_i$) and ability ($Ability_i$)
      		  \begin{equation}
        			Y_{i}=\beta_0+\beta_{1}Ed_{i} +\beta_{2}Ability_{i} +\epsilon
       		 \end{equation}
		     \pause
		 \item What happens if we just estimate our original model, omitting ability?
      		  \begin{equation}
        			Y_{i}=\gamma_0+\gamma_{1}Ed_{i} +\mu
       		 \end{equation}		
		     \pause
		\item Using our formula for a regression coefficient  $E(\hat{\gamma_1})=\frac{cov(Y,Ed_i)}{var(Ed_i)}$, and substituting $\alpha+\beta_{1}Ed_{i} +\beta_{2}Ability_{i} +\epsilon$ for $Y_i$, we get:
		 \begin{equation}
        			E(\hat{\gamma_1})=\frac{cov(\alpha+\beta_{1}Ed_{i} +\beta_{2}Ability_{i} +\epsilon,Ed_i)}{var(Ed_i)}
       		 \end{equation}	

	\end{itemize}

\end{frame}
\begin{frame}{Omitted Variable Bias}
	\begin{equation}
        		=\frac{cov(\beta_0,Ed_i)}{var(Ed_i)}+\beta_1 +\beta_{2}\frac{cov(Ability_i,Ed_i)}{var(Ed_i)}+\frac{cov(\epsilon, Ed_i)}{var(Ed_i)}
       	\end{equation}	

	\begin{itemize}
		\item The first term is zero since $\beta_0$ is a constant and the last term is also zero given the main OLS assumptions ($cov(\epsilon, X_i)=0$)
		\pause
		\item This leaves us with the omitted variable bias formula:
	\begin{equation}
        		E(\hat{\gamma_1})=+\beta_1 +\beta_{2}\frac{cov(Ability_i,Ed_i)}{var(Ed_i)}
       	\end{equation}	
	\pause	
	\item Note that the last term $\frac{cov(Ability_i,Ed_i)}{var(Ed_i)}$ is the coefficient from a bivariate regression of $Ability_i$ on $Ed_i$
	\item OVB is driven by factor(s) that are more correlated with your independent variables of interest and your dependent variable. 
	\end{itemize}
\end{frame}

\begin{frame}[<+->]{Approaches to Estimating the Returns to Education}
\begin{itemize}
\item There are (roughly) six approaches to estimating the returns: \\~\\
	\begin{enumerate}
	\item Bound the returns to education \\~\\
	\item Selection-on-observables \\~\\
	\item Structural models \\~\\
	\item Twin studies \\~\\
	\item Quasi-experimental evidence \\~\\
	\item Random Control Trials (RCTs)
	\end{enumerate}
\end{itemize}
\end{frame}


\end{document}